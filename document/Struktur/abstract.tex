\chapter*{Abstract}
This thesis presents a way to benchmark typical \textit{Node.js} applications. Node.js is a \textit{JavaScript} based runtime environment mostly for server-sided web applications. There are already existing JavaScript benchmark suites which just test the performance of the core runtime. This thesis shows how to test the performance of a typical Node.js application with its typical characteristics, such as request handling, I/O performance and VM-level performance. 

Firstly, it evaluates which packages are currently the most popular ones by analyzing the registry of the package manager \textit{npm}.
Then some of these popular packages which are considered as relevant for performance testing are used in some benchmark applications. The thesis also describes how these applications are benchmarked with \textit{Apache JMeter}.



\newpage

\chapter*{Kurzfassung}
Diese Bachelorarbeit pr\"asentiert eine M\"oglichkeit, typische \textit{Node.js} Applikationen zu benchmarken. Node.js ist eine auf \textit{JavaScript} basierende Plattform die haupts\"achlich f\"ur serverseitige Webapplikationen verwendet wird. Es gibt bereits einige JavaScript Benchmark Suiten, welche aber nur das Kern-Laufzeitverhalten testen. Diese Arbeit zeigt wie man die Performance einer typischen Node.js Applikation und ihrer typischen Charakteristiken, wie zum Beispiel das Request-Handling, die I/O Performance und die VM-Level Performance, testen kann.

Zuerst werden die derzeit popul\"arsten Pakete evaluiert, indem das Verzeichnis des Paket-Managers \textit{npm} analysiert wird. Dann werden einige dieser Pakete, welche f\"ur das durchf\"uhren eines Benchmarks relevant sind, in verschiedenen Test-Applikationen verwendet. Weiters zeigt diese Bachelorarbeit, wie man diese Applikationen mit \textit{Apache JMeter} testen und auswerten kann.


\newpage