\documentclass[11pt,a4paper,austrian,titlepage,
chapterprefix,headsepline,parskip,pdftex,
,numbers=noenddot,bibliography=totoc]{scrreprt}

%%% old: \documentclass[11pt,a4paper,austrian,titlepage,
%chapterprefix,headsepline,parskip,pdftex,
%,pointlessnumbers,bibtotoc]{scrreprt}

%%% Abs‰tze bei tieferen Ebenen einschalten
\makeatletter %% Sonderbedeutung von @ aufheben
\renewcommand{\paragraph}{\@startsection
   {paragraph} % name
   {4} % ebene
   {0mm} % einzug
   {-\baselineskip} % vorabstand
   {0.1\baselineskip} % nachabstand
   {\normalfont\normalsize\bfseries}} % stil
\makeatother %% Sonderbedeutung von @ wieder

\makeatletter %% Sonderbedeutung von @ aufheben
\renewcommand{\subparagraph}{\@startsection
   {subparagraph} % name
   {5} % ebene
   {0mm} % einzug
   {-\baselineskip} % vorabstand
   {0.1\baselineskip} % nachabstand
   {\normalfont\normalsize\bfseries}} % stil
\makeatother %% Sonderbedeutung von @ wieder

\usepackage{setspace}
\onehalfspacing

\usepackage[pdftex]{graphicx}

% for colours
\usepackage[pdftex]{color}

\usepackage[colorlinks=true,
    linkcolor=black,
    citecolor=black,
%    pagecolor=black,
    urlcolor=black,
    breaklinks=true,
    bookmarksnumbered=true,
    hypertexnames=false,
    pdfpagemode=UseOutlines,
    pdfview=FitH,
    plainpages=false,
    pdfpagelabels,
    bookmarks=true,
    linktocpage=true]{hyperref}

\hypersetup{pdfauthor={Christopher Warmbold},
    pdftitle={Node.js benchmark suite},
    pdfsubject={Bachelor Thesis},
    pdfkeywords={},
    pdfcreator={pdfLaTeX with hyperref (\today})}

%%% f¸r Source-Code
\usepackage{listings}
\lstset{language=Java}


% Interner-Link-Befehl
\newcommand{\internerLink}[1]{\hyperref[#1]
{Siehe \ref*{#1}~\nameref{#1} auf S.~\pageref{#1}}}

% Interner-Link-Befehl 2
\newcommand{\ffinternerLink}[1]{\hyperref[#1]
{Siehe S.~\pageref{#1}ff}}

% Interner-Link-Befehl x
\newcommand{\xinternerLink}[1]{\hyperref[#1]
{\ref*{#1}~\nameref{#1} auf S.~\pageref{#1}}}


%%% continous footnote
\newcounter{cfootnotecounter}
\newcommand{\cfootnote}[1]{\stepcounter{cfootnotecounter}
\footnote[\value{cfootnotecounter}]{#1}}

\flushbottom

\usepackage{geometry}
\geometry{a4paper,left=20mm, right=20mm, top=20mm, bottom=20mm, includeheadfoot}
% change page settings
	%\setlength{\hoffset}{0mm} \setlength{\voffset}{0mm}
	%\setlength{\evensidemargin}{14.6mm}
	%\setlength{\oddsidemargin}{14.6mm} \setlength{\topmargin}{-20mm}
	%\setlength{\headheight}{15mm} \setlength{\headsep}{9mm}
	%\setlength{\textheight}{242mm} \setlength{\textwidth}{145mm}
	%\setlength{\footskip}{10mm}
%%% Nachfolgendes nicht notwendig wg. Klassenoption parskip
%\setlength{\parskip}{3ex plus0.5ex minus0.5ex}
%\setlength{\parindent}{0mm}

%%% Abst‰nde von float-Umgebungen
%\setlength{\textfloatsep}{25pt plus5pt minus5pt}
%\setlength{\intextsep}{25pt plus5pt minus5pt}

%%% Gliederungs-Nummern in den Rand schreiben
%\renewcommand*{\othersectionlevelsformat}[1]{%
%\llap{\csname the#1\endcsname\autodot\enskip}}

%%% In Kopfzeile nur Kapitel-Text ohne "Kapitel x"
\renewcommand*{\chaptermarkformat}{}

%%% Formatierung von chapter ‰ndern
\setkomafont{chapter}{\Huge}
\renewcommand*{\chapterformat}
{\LARGE{\chapappifchapterprefix{\ }
\thechapter\autodot\enskip}}

%%% Kopfzeile
\usepackage[automark]{scrpage2}

\clearscrheadings \clearscrplain \clearscrheadfoot
\pagestyle{scrheadings}
\ohead{\pagemark}
\ihead{\headmark}
\cfoot{}

%%% Formatierung von Kapitel-Seiten
\renewcommand*{\chapterpagestyle}{scrheadings}

%% Gliederung TOC und Nummerierungstiefe
\setcounter{tocdepth}{\subsubsectionlevel}
\setcounter{secnumdepth}{\paragraphlevel}

%%% Array f¸r Tabellen
\usepackage{array}

%%% Schriftarten
\addtokomafont{chapter}{\sffamily}
\addtokomafont{sectioning}{\rmfamily}

% Sprache
 %\usepackage[naustrian ,ngerman]{babel} 
 \usepackage[english]{babel} 
% Eingabe von Umlauten
\usepackage[applemac]{inputenc}
% Verwenden von T1 Fonts
\usepackage[T1]{fontenc}
% \usepackage{ae}
\usepackage{lmodern}

% URLs
\usepackage{url}

%%% Einbinden von kompletten PDF-Seiten
\usepackage{pdfpages}

\usepackage{algorithmic}
\usepackage{algorithm}

\usepackage{scrhack} 
\usepackage{longtable}